% Options for packages loaded elsewhere
\PassOptionsToPackage{unicode}{hyperref}
\PassOptionsToPackage{hyphens}{url}
\PassOptionsToPackage{dvipsnames,svgnames*,x11names*}{xcolor}
%
\documentclass[
]{article}
\usepackage{lmodern}
\usepackage{amsmath}
\usepackage{ifxetex,ifluatex}
\ifnum 0\ifxetex 1\fi\ifluatex 1\fi=0 % if pdftex
  \usepackage[T1]{fontenc}
  \usepackage[utf8]{inputenc}
  \usepackage{textcomp} % provide euro and other symbols
  \usepackage{amssymb}
\else % if luatex or xetex
  \usepackage{unicode-math}
  \defaultfontfeatures{Scale=MatchLowercase}
  \defaultfontfeatures[\rmfamily]{Ligatures=TeX,Scale=1}
\fi
% Use upquote if available, for straight quotes in verbatim environments
\IfFileExists{upquote.sty}{\usepackage{upquote}}{}
\IfFileExists{microtype.sty}{% use microtype if available
  \usepackage[]{microtype}
  \UseMicrotypeSet[protrusion]{basicmath} % disable protrusion for tt fonts
}{}
\makeatletter
\@ifundefined{KOMAClassName}{% if non-KOMA class
  \IfFileExists{parskip.sty}{%
    \usepackage{parskip}
  }{% else
    \setlength{\parindent}{0pt}
    \setlength{\parskip}{6pt plus 2pt minus 1pt}}
}{% if KOMA class
  \KOMAoptions{parskip=half}}
\makeatother
\usepackage{xcolor}
\IfFileExists{xurl.sty}{\usepackage{xurl}}{} % add URL line breaks if available
\IfFileExists{bookmark.sty}{\usepackage{bookmark}}{\usepackage{hyperref}}
\hypersetup{
  pdftitle={MSDS 7333 Spring 2021: Case Study 04},
  pdfauthor={Sachin Chavan,Tazeb Abera,Gautam Kapila,Sandesh Ojha},
  colorlinks=true,
  linkcolor=Maroon,
  filecolor=Maroon,
  citecolor=Blue,
  urlcolor=gray,
  pdfcreator={LaTeX via pandoc}}
\urlstyle{same} % disable monospaced font for URLs
\usepackage[margin=1in]{geometry}
\usepackage{graphicx}
\makeatletter
\def\maxwidth{\ifdim\Gin@nat@width>\linewidth\linewidth\else\Gin@nat@width\fi}
\def\maxheight{\ifdim\Gin@nat@height>\textheight\textheight\else\Gin@nat@height\fi}
\makeatother
% Scale images if necessary, so that they will not overflow the page
% margins by default, and it is still possible to overwrite the defaults
% using explicit options in \includegraphics[width, height, ...]{}
\setkeys{Gin}{width=\maxwidth,height=\maxheight,keepaspectratio}
% Set default figure placement to htbp
\makeatletter
\def\fps@figure{htbp}
\makeatother
\setlength{\emergencystretch}{3em} % prevent overfull lines
\providecommand{\tightlist}{%
  \setlength{\itemsep}{0pt}\setlength{\parskip}{0pt}}
\setcounter{secnumdepth}{-\maxdimen} % remove section numbering
\usepackage{siunitx}
\newcolumntype{d}{S[table-format=3.2]}
\usepackage{multicol}
\usepackage{float}
\usepackage{fancyhdr}
\pagestyle{fancy}
\fancyhead[L]{Case Study 03}
\usepackage{float}
\ifluatex
  \usepackage{selnolig}  % disable illegal ligatures
\fi

\title{MSDS 7333 Spring 2021: Case Study 04}
\usepackage{etoolbox}
\makeatletter
\providecommand{\subtitle}[1]{% add subtitle to \maketitle
  \apptocmd{\@title}{\par {\large #1 \par}}{}{}
}
\makeatother
\subtitle{Influenza forecast}
\author{Sachin Chavan,Tazeb Abera,Gautam Kapila,Sandesh Ojha}
\date{2021 February 20}

\begin{document}
\maketitle

\hypertarget{introduction}{%
\section{Introduction}\label{introduction}}

\textbf{Influenza} most commonly known as \textbf{flu} is an infectious
respiratory disease caused by virus. Virus was named as influenza and
exist in four different types. Type A , Type B, Type C and Type D. Based
on historical data it occurs in winter or in monsoon season. There
several studies have been conducted on the causes of occurring in the
specific season and one of the study has linked it to vitamin D levels
in human \(^{[1]}\). It's during winter or rainy season when people stay
mostly indoors and are less expose to sun and levels of vitamin D falls.
Staying inside homes has also been seen as one of the reason of
transmission the disease as people are in close contact of each other.
That explains a bit of seasonal occurrence of the flu every year. These
viruses infect nose, throat and lungs and leads to mild to severe
illness and has been observed that it can also lead to death of proper
attention is not given.

Data shows that globally on an average 389K deaths occurs due to flu and
it mostly affect people who are above 60 years of age\(^{[2]}\). In the
United States alone more than 200K hospitalization are due to Influenza
every year\(^{[3]}\).
\href{Centers\%20for\%20Disease\%20Control\%20and\%20Prevention}{Centers
for Disease Control and Prevention} has reported range of death is
between 3500 and 49K every year\(^{[4]}\). Out of many strains of the
viruses few have potential to create pandemics/epidemics. One the
deadliest pandemic occurred in year 1918 which infected 33\% of world
population and caused 100 million deaths worldwide. Other than that it
has been observed that Pandemic occurs 3 times in century.

\newpage

\hypertarget{business-understanding}{%
\section{Business Understanding}\label{business-understanding}}

\newpage

\hypertarget{data-extraction-evaluation}{%
\section{Data Extraction \&
Evaluation}\label{data-extraction-evaluation}}

\newpage

\hypertarget{modeling-building}{%
\section{Modeling Building}\label{modeling-building}}

\newpage

\hypertarget{forecasting}{%
\section{Forecasting}\label{forecasting}}

\newpage

\hypertarget{conclusions}{%
\section{Conclusions}\label{conclusions}}

\newpage

\hypertarget{references}{%
\section{References}\label{references}}

\begin{enumerate}
\def\labelenumi{\arabic{enumi}.}
\tightlist
\item
  Cannell JJ, Vieth R, Umhau JC, Holick MF, Grant WB, Madronich S, et
  al.~(December 2006). ``Epidemic influenza and vitamin D''.
  Epidemiology and Infection. 134 (6): 1129--40.
  \url{doi:10.1017/S0950268806007175}. PMC 2870528. PMID 16959053.
\item
  Paget J, Spreeuwenberg P, Charu V, Taylor RJ, Iuliano AD, Bresee J, et
  al.~(December 2019). ``Global mortality associated with seasonal
  influenza epidemics: New burden estimates and predictors from the
  GLaMOR Project''. Journal of Global Health. 9 (2): 020421.
  \url{doi:10.7189/jogh.09.020421}. PMC 6815659. PMID 31673337.
\item
  Thompson WW, Shay DK, Weintraub E, Brammer L, Cox N, Anderson LJ, et
  al.~(January 2003). ``Mortality associated with influenza and
  respiratory syncytial virus in the United States''. JAMA. 289 (2):
  179--86. \url{doi:10.1001/jama.289.2.179}. PMID 12517228. S2CID
  5018362.
\end{enumerate}

\end{document}
